\documentclass{article}
\usepackage[utf8]{inputenc}
\usepackage[a4paper, total={6.5in, 10in}]{geometry}
\usepackage{amsmath}
\usepackage{enumerate}
\usepackage{amssymb}
\usepackage{tikz}
\usepackage[backend=biber,style=authoryear]{biblatex}
\addbibresource{src/references.bib}

\title{Logic Coursework}
\author{Michal Pluta}
\date{February 2023}

\newcommand{\True}{\textnormal{T}}
\newcommand{\False}{\textnormal{F}}
\newcommand{\predS}{\hspace{0.05cm} \textnormal{S}}
\newcommand{\predicate}{\predS (x,y,z)}
\newcommand{\impgap}{\hspace{0.05cm}}
\newcommand{\imp}{\impgap \longrightarrow \impgap}

% From https://tex.stackexchange.com/questions/47063/rightarrow-vs-implies-and-does-not-imply-symbol
\newcommand{\notimplies}{%
  \mathrel{{\ooalign{\hidewidth$\not\phantom{=}$\hidewidth\cr$\imp$}}}}

\begin{document}

\maketitle

\begin{enumerate}
    % ------------------------------------------------------------
    % Question 1
    % ------------------------------------------------------------
    \item
    {
    To prove a set of logical connectives $F_1$ is functionally complete, you have to show that all the connectives in an already proven complete set of logical connectives $F_2$ can be expressed using the connectives in $F_1$.

    For these questions, I will use $F_2 = \{\lnot, \land, \lor\}$.

    The reason why $\{\lnot, \land, \lor\}$ is functionally complete is because $\{\lnot, \land, \lor\, \rightarrow, \leftrightarrow \}$ is functionally complete and $\{ \rightarrow, \leftrightarrow \}$ can be expressed as:
    \begin{align*}
        \varphi \rightarrow \psi &\equiv \lnot \varphi \lor \psi \\
        \varphi \leftrightarrow \psi &\equiv (\varphi \land \psi) \lor (\lnot \varphi \land \lnot \psi)
    \end{align*}
    } \\

    i) \\
    {
    In this case, 
    $$F_1 = \{\lnot, \lor\},$$
    
    and all that needs to be shown is that $\land$ can be expressed using the connectives in $F_1$
    $$\varphi \land \psi \equiv \lnot(\lnot \varphi \lor \lnot \psi)$$

    this is simply an application of \textit{De Morgan's Law} and it shows that the set $F_1$ is functionally complete
    } \\ \\

    ii) \\
    {
    In this case,
    $$F_1 = \{\rightarrow, 0\},$$

    by showing that

    \begin{center}\begin{tabular}{ccl}\
    $\lnot \varphi$ & $\equiv$ & $\varphi \rightarrow 0$ \\
    
    $\varphi \lor \psi$ & $\equiv$ & $(\varphi \rightarrow 0) \rightarrow \psi$ \\
    
    $\varphi \land \psi$ & $\equiv$ & $(((\varphi \rightarrow 0) \rightarrow 0) \rightarrow (\psi \rightarrow 0)) \rightarrow 0$
    \end{tabular}\end{center}
    we show that all the logical connectives in $F_2$ can be expressed using the logical connectives in $F_1$ 
    
    $\therefore F_1$ is functionally complete
    } \\ \\

    iii) \\
    {
    In this case,
    $$F_1 = \{\uparrow, \lor\},$$

    by showing that

    \begin{center}\begin{tabular}{ccl}\
    $\lnot \varphi$ & $\equiv$ & $\varphi \uparrow \varphi$ \\
    
    $\varphi \land \psi$ & $\equiv$ & $(\varphi \uparrow \psi) \uparrow (\varphi \uparrow \psi)$
    \end{tabular}\end{center}
    we show that all the logical connectives in $F_2$ can be expressed using the logical connectives in $F_1$
    
    $\therefore F_1$ is functionally complete
    } \\ \\

    \pagebreak

    iv)
    {
    \begin{center}
    $\{\rightarrow, \longleftrightarrow\}$ is not functionally complete
    \end{center}
   
    This is because you cannot construct the logical operator $\lnot$ from the connectives in $\{\rightarrow, \longleftrightarrow\}$.
   
    More specifically, $\rightarrow$ and $\longleftrightarrow$ are both truth-preserving connectives. This means that they have true outputs whenever the inputs are true therefore it is impossible for the $\lnot$ operator to be constructed.
    } \\ \\

    \pagebreak

    % ------------------------------------------------------------
    % Question 2
    % ------------------------------------------------------------
    \item
    i) \\ \\
    {
    To show that:
    $$\forall x \exists y \forall z \predicate \imp \lnot\exists x \forall y \exists z \lnot \predicate$$
    we can use a deductive proof as follows
    
    \begin{center}\begin{tabular}{ccl}\
    $\forall x \exists y \forall z \predicate$ & $\equiv$ & $\lnot \lnot(\forall x \exists y \forall z \predicate)$ \\
    & $\equiv$ & $\lnot (\lnot \forall x \exists y \forall z \predicate)$ \\
    & $\equiv$ & $\lnot(\exists x (\lnot \exists y \forall z \predicate))$ \\
    & $\equiv$ & $\lnot(\exists x (\forall y (\lnot \forall z \predicate)))$ \\
    & $\equiv$ & $\lnot(\exists x (\forall y (\exists z (\lnot \predicate))))$ \\
    & $\equiv$ & $\lnot\exists x \forall y \exists z \lnot \predicate$ \\
    \end{tabular}\end{center}
    Therefore the statement $\forall x \exists y \forall z \predicate \imp \lnot\exists x \forall y \exists z \lnot \predicate$
    } \\ \\

    ii) \\ \\
    {
    To disprove:
    $$\forall x \exists y \forall z \predicate \imp \exists y \forall x \forall z \predicate$$
    we can use a counterexample: \\

    Let the domain be $\{0, 1\}$ and the model be $\{(0,1,0), (0,1,1), (1,0,0), (1,0,1)\}$
    }
    {
    \begin{center}\begin{tabular}{ccl}\
    $\forall x \exists y \forall z \predicate$ & $\equiv$ & $\exists y \forall z \predS(0,y,z) \land \exists y \forall z \predS(1,y,z)$ \\ \\
    
    & $\equiv$ & $(\forall z \predS(0,0,z) \lor \forall z \predS(0,1,z)) \land (\forall z \predS(1,0,z) \lor \forall z \predS(1,1,z))$ \\ \\
    
    & $\equiv$ & $((\predS(0,0,0) \land \predS(0,0,1)) \lor (\predS(0,1,0) \land \predS(0,1,1))) \hspace{0.1cm} \land$ \\
    & & $((\predS(1,0,0) \land \predS(1,0,1)) \lor (\predS(1,1,0) \land \predS(1,1,1)))$ \\ \\
    
    & $\equiv$ & $((\False \land \False) \lor (\True \land \True)) \land ((\True \land \True) \lor (\False \land \False))$ \\ \\
    
    & $\equiv$ & $(\False \lor \True) \land (\True \lor \False)$ \\ \\
    
    & $\equiv$ & $\True \land \True$ \\ \\
    
    & $\equiv$ & $\True$ \\ 
    \end{tabular}\end{center}
    }
    {
    %
    \begin{center}\begin{tabular}{ccl}\
    $\exists y \forall x \forall z \predicate$ & $\equiv$ & $\forall x \forall z \predS(x,0,z) \lor \forall x \forall z \predS(x,1,z)$ \\ \\
    
    & $\equiv$ & $(\forall z \predS(0,0,z) \land \forall z \predS(1,0,z)) \lor (\forall z \predS(0,1,z) \land \forall z \predS(1,1,z))$ \\ \\
    
    & $\equiv$ & $((\predS(0,0,0) \land \predS(0,0,1)) \land (\predS(1,0,0) \land \predS(1,0,1))) \hspace{0.1cm} \lor$ \\
    & & $((\predS(0,1,0) \land \predS(0,1,1)) \land (\predS(1,1,0) \land \predS(1,1,1)))$ \\ \\
    
    & $\equiv$ & $((\False \land \False) \land (\True \land \True)) \lor ((\True \land \True) \land (\False \land \False))$ \\ \\
    
    & $\equiv$ & $(\False \land \True) \lor (\True \land \False)$ \\ \\
    
    & $\equiv$ & $\False \lor \False$ \\ \\
    
    & $\equiv$ & $\False$ \\ 
    \end{tabular}\end{center}

    Using this model, we have shown that $\True \imp \False$ which is a contradiction hence the original statement is wrong.
    
    $$\forall x \exists y \forall z \predicate \notimplies \exists y \forall x \forall z \predicate$$
    }
    \pagebreak

    iii) \\ \\
    {
    To prove
    $$\exists y \forall x \forall z \predicate \imp \forall x \exists y \forall z \predicate$$

    one could use the laws of quantifier independence i.e.
    $$\textbf{Law 8: }(\exists x)(\forall y) \varphi(x,y) \imp (\forall y)(\exists x) \varphi(x,y)$$ 
    \begin{flushright}
        \autocite{hammond}
    \end{flushright}
    }
    {
    However, another way to approach this is to use an intuitive approach. \\

    Let $D$ be any non-empty domain.
    
    Suppose $\exists y=k$ where $k \in D$.
    
    Let $m \in D$ and $n \in D$.

    For $\exists y \forall x \forall z \predicate$ to be true, the model must contain all possible $(k,m,n)$. \\

    {
    \hspace{0.5cm} e.g. if $D = \{1, -1\}$ and $y=1$,
    
    \hspace{0.5cm} then the model must contain $(1, 1, 1), (1, 1, -1), (-1, 1, 1),$ and $(-1, 1, -1)$.
    } \\

    What this also means is that, for every possible value of $x$, there exists a value of $y$, namely $y=k$, such that it is true for every value of $z$. \\

    {
    \hspace{0.5cm} Using the previous example with $D = \{1, -1\}$ and $y=1$,
    
    \hspace{0.5cm} For every possible value of $x$, there exists a value of $y$, namely $y=k$ for every value of $z$.

    \hspace{0.5cm} i.e. There are $|D| = 2$ possible values for $x$
    
    \hspace{1.13cm} If $x=1$, $(1,1,1)$ and $(1,1,-1)$ exist, so there exists a $y$ $(=1)$ for all $z$
    
    \hspace{1.13cm} If $x=-1$, $(-1,1,1)$ and $(-1,1,-1)$ exist, so there exists a $y$ $(=1)$ for all $z$ \\

    \hspace{0.5cm} $\therefore \forall x \exists y \forall z \predicate$.
    }
    
    }

    \pagebreak

    iv) \\ \\
    {
    To disprove:
    $$\exists y \forall x \forall z \lnot \predicate \imp \lnot \forall x \exists y \forall z \predicate$$
    we can use a counterexample: \\

    Let the domain be $\{0, 1\}$ and the model be $\{(0,1,0), (0,1,1), (1,1,0), (1,1,1)\}$
    }
    {
    \begin{center}\begin{tabular}{ccl}\
   $\exists y \forall x \forall z \lnot \predicate$ & $\equiv$ & $\forall x \forall z \lnot \predS(x,0,z) \lor \forall x \forall z \lnot \predS(x,1,z)$ \\ \\
    
    & $\equiv$ & $(\forall z \lnot \predS(0,0,z) \land \forall z \lnot \predS(1,0,z)) \lor (\forall z \lnot \predS(0,1,z) \land \forall z \lnot \predS(1,1,z))$ \\ \\
    
    & $\equiv$ & $((\lnot \predS(0,0,0) \land \lnot \predS(0,0,1)) \land (\lnot \predS(1,0,0) \land \lnot \predS(1,0,1))) \hspace{0.1cm} \lor$ \\
    & & $((\lnot \predS(0,1,0) \land \lnot \predS(0,1,1)) \land (\lnot \predS(1,1,0) \land \lnot \predS(1,1,1)))$ \\ \\
    
    & $\equiv$ & $((\True \land \True) \land (\True \land \True)) \lor ((\False \land \False) \land (\False \land \False))$ \\ \\
    
    & $\equiv$ & $(\True \land \True) \lor (\False \land \False)$ \\ \\
    
    & $\equiv$ & $\True \lor \False$ \\ \\
    
    & $\equiv$ & $\True$ \\ 
    \end{tabular}\end{center}
    }
    {
    %
    \begin{center}\begin{tabular}{ccl}\
    $\lnot \forall x \exists y \forall z \predicate$ & $\equiv$ & $\lnot (\exists y \forall z \predS(0,y,z) \land \exists y \forall z \predS(1,y,z))$ \\ \\
    
    & $\equiv$ & $\lnot ((\forall z \predS(0,0,z) \lor \forall z \predS(0,1,z)) \land (\forall z \predS(1,0,z) \lor \forall z \predS(1,1,z))) \hspace{0.42cm}$ \\ \\
    
    & $\equiv$ & $\lnot (((\predS(0,0,0) \land \predS(0,0,1)) \lor (\predS(0,1,0) \land \predS(0,1,1))) \hspace{0.1cm} \land$ \\
    & & $\hspace{0.235cm} ((\predS(1,0,0) \land \predS(1,0,1)) \lor (\predS(1,1,0) \land \predS(1,1,1)))) \hspace{0.42cm}$ \\ \\
    
    & $\equiv$ & $\lnot (((\False \land \False) \lor (\True \land \True)) \land ((\False \land \False) \lor (\True \land \True)))$ \\ \\
    
    & $\equiv$ & $\lnot ((\False \lor \True) \land (\False \lor \True))$ \\ \\
    
    & $\equiv$ & $\lnot (\True \land \True)$ \\ \\
    
    & $\equiv$ & $\lnot (\True)$ \\ \\

    & $\equiv$ & $\False$ \\ 
    \end{tabular}\end{center}

    Using this model, we have shown that $\True \imp \False$ which is a contradiction hence the original statement is wrong.
    
    $$\exists y \forall x \forall z \lnot \predicate \notimplies \lnot \forall x \exists y \forall z \predicate$$
    }
    \pagebreak

    % ------------------------------------------------------------
    % Question 3
    % ------------------------------------------------------------
    \item
    {
    There are two main ways to approach these questions
    \begin{enumerate}[1.]
        \item
        Constructing a visual aid in the form of a parse tree which represents the predicate statement, and then going up the parse tree checking if the elements in the domain satisfy each branch
        \item
        Representing the predicate statement using propositional logic, and then evaluating statement.
    \end{enumerate}
    For these questions, I will use a combination of both. \\

    Domain = $\{0, 1\}$ \\
    A dashed line represents the existential quantifier \\
    A solid line represents the universal quantifier
    } \\
    
    (i) 
    {
    Evaluate $\forall x \exists y \forall z \hspace{0.05cm} \textnormal{S}(x,y,z)$ on $\{(0,1,0), (1,0,1), (0,1,1), (1,0,0)\}$
    } \\
    {
    \begin{center}\begin{tikzpicture}
    \node {$\forall x$} [
        level 1/.style={sibling distance=8cm},
        level 2/.style={sibling distance=4cm}, 
        level 3/.style={sibling distance=4cm},
        level 4/.style={sibling distance=2cm}
    ]
        child {node {$x=0$}
            child {node {$\exists y$}
                child {node {$y=0$}
                    child {node {$\forall z$}
                        child {node [label=below:$(0\,0\,0)$] {$z=0$}}
                        child {node [label=below:$(0\,0\,1)$] {$z=1$}}
                        edge from parent [solid]
                    } edge from parent [black, dashed]
                } 
                child {node {$y=1$}
                    child {node {$\forall z$}
                         child {node [label=below:\textcolor{green}{$(0\,1\,0)$}] {$z=0$}}
                         child {node [label=below:\textcolor{green}{$(0\,1\,1)$}] {$z=1$}}
                         edge from parent [solid]
                    }
                    edge from parent [dashed]
                }
            } edge from parent [green]
        }
        child {node {$x=1$}
            child {node {$\exists y$}
                child {node {$y=0$}
                    child {node {$\forall z$} 
                        child {node [label=below:\textcolor{green}{$(1\,0\,0)$}] {$z=0$}}
                        child {node [label=below:\textcolor{green}{$(1\,0\,1)$}] {$z=1$}}
                        edge from parent [solid]
                    } edge from parent [dashed]
                }
                child {node {$y=1$}
                    child {node {$\forall z$}
                        child {node [label=below:$(1\,1\,0)$] {$z=0$}}
                        child {node [label=below:$(1\,1\,1)$] {$z=1$}}
                        edge from parent [solid]
                    } edge from parent [black, dashed]
                }
            } edge from parent [green]
        };
    \end{tikzpicture}
    Figure 1: Visual representation of $\forall x \exists y \forall z \predicate$
    
    $ $
    \end{center}
    }
    {
    \begin{align*}
    \forall x \exists y \forall z \predicate &\equiv \exists y \forall z \predS(0,y,z) \land \exists y \forall z \predS(1,y,z) \\ \\
    &\equiv (\forall z \predS(0,0,z) \lor \forall z \predS(0,1,z)) \land (\forall z \predS(1,0,z) \lor \forall z \predS(1,1,z)) \\ \\
    &\equiv ((\predS(0,0,0) \land \predS(0,0,1)) \lor (\predS(0,1,0) \land \predS(0,1,1))) \hspace{0.1cm} \land \\
    & \hspace{0.48cm} ((\predS(1,0,0) \land \predS(1,0,1)) \lor (\predS(1,1,0) \land \predS(1,1,1))) \\ \\
    &\equiv ((\False \land \False) \lor (\True \land \True)) \land ((\True \land \True) \lor (\False \land \False)) \\ \\
    &\equiv (\False \lor \True) \land (\True \lor \False)\\ \\
    &\equiv \True \land \True \\ \\
    &\equiv \True \\ 
    \end{align*}
    This shows that the sentence is true on the provided model.
    }

    \pagebreak

    (ii) 
    {
    Evaluate $\forall x \exists y \forall z \hspace{0.05cm} \textnormal{S}(x,y,z)$ on $\{(0,1,1), (1,0,0), (0,1,0), (0,0,1), (0,0,0)\}$
    } \\
    {
    \begin{center}\begin{tikzpicture}
    \node {$\forall x$} [
        level 1/.style={sibling distance=8cm},
        level 2/.style={sibling distance=4cm}, 
        level 3/.style={sibling distance=4cm},
        level 4/.style={sibling distance=2cm}
    ]
        child {node {$x=0$}
            child {node {$\exists y$}
                child {node {$y=0$}
                    child {node {$\forall z$}
                        child {node [label=below:\textcolor{green}{$(0\,0\,0)$}] {$z=0$}}
                        child {node [label=below:\textcolor{green}{$(0\,0\,1)$}] {$z=1$}}
                        edge from parent [solid]
                    } edge from parent [dashed]
                } 
                child {node {$y=1$}
                    child {node {$\forall z$}
                         child {node [label=below:\textcolor{green}{$(0\,1\,0)$}] {$z=0$}}
                         child {node [label=below:\textcolor{green}{$(0\,1\,1)$}] {$z=1$}}
                         edge from parent [solid]
                    }
                    edge from parent [dashed]
                }
            } edge from parent [green]
        }
        child {node {$x=1$}
            child {node {$\exists y$}
                child {node {$y=0$}
                    child {node {$\forall z$} 
                        child {node [label=below:\textcolor{green}{$(1\,0\,0)$}] {$z=0$} edge from parent [green]}
                        child {node [label=below:$(1\,0\,1)$] {$z=1$}}
                        edge from parent [solid]
                    } edge from parent [dashed]
                }
                child {node {$y=1$}
                    child {node {$\forall z$}
                        child {node [label=below:$(1\,1\,0)$] {$z=0$}}
                        child {node [label=below:$(1\,1\,1)$] {$z=1$}}
                        edge from parent [solid]
                    } edge from parent [dashed]
                }
            } edge from parent [red]
        };
    \end{tikzpicture}
    Figure 1: Visual representation of $\forall x \exists y \forall z \predicate$
    
    $ $
    \end{center}
    }
    {
    \begin{center}\begin{tabular}{ccl}\
    $\forall x \exists y \forall z \predicate$ & $\equiv$ & $\exists y \forall z \predS(0,y,z) \land \exists y \forall z \predS(1,y,z)$ \\ \\
    
    & $\equiv$ & $(\forall z \predS(0,0,z) \lor \forall z \predS(0,1,z)) \land (\forall z \predS(1,0,z) \lor \forall z \predS(1,1,z))$ \\ \\
    
    & $\equiv$ & $((\predS(0,0,0) \land \predS(0,0,1)) \lor (\predS(0,1,0) \land \predS(0,1,1))) \hspace{0.1cm} \land$ \\
    & & $((\predS(1,0,0) \land \predS(1,0,1)) \lor (\predS(1,1,0) \land \predS(1,1,1)))$ \\ \\
    
    & $\equiv$ & $((\True \land \True) \lor (\True \land \True)) \land ((\True \land \False) \lor (\False \land \False))$ \\ \\
    
    & $\equiv$ & $(\True \lor \True) \land (\False \lor \False)$ \\ \\
    
    & $\equiv$ & $\True \land \False$ \\ \\
    
    & $\equiv$ & $\False$ \\ 
    \end{tabular}\end{center}
    This shows that the sentence is false on the provided model.
    }

    \pagebreak

    (iii) 
    {
    Evaluate $\exists y \forall x \exists z \predicate$ on $\{(1,0,0), (0,0,1), (1,1,1)\}$
    } \\
    {
    \begin{center}\begin{tikzpicture}
    \node {$\exists y$} [
        level 1/.style={sibling distance=8cm},
        level 2/.style={sibling distance=4cm}, 
        level 3/.style={sibling distance=4cm},
        level 4/.style={sibling distance=2cm}
    ]
        child {node {$y=0$}
            child {node {$\forall x$}
                child {node {$x=0$}
                    child {node {$\exists z$}
                        child {node [label=below:$(0\,0\,0)$] {$z=0$}
                        edge from parent [dashed, black]}
                        child {node [label=below:\textcolor{green}{$(0\,0\,1)$}] {$z=1$}
                        edge from parent [dashed]}
                    }
                } 
                child {node {$x=1$}
                    child {node {$\exists z$}
                         child {node [label=below:\textcolor{green}{$(1\,0\,0)$}] {$z=0$}
                         edge from parent [dashed]}
                         child {node [label=below:$(1\,0\,1)$] {$z=1$}
                         edge from parent [dashed, black]}
                    }
                }
                edge from parent [solid]
            } edge from parent [dashed, green]
        }
        child {node {$y=1$}
            child {node {$\forall x$}
                child {node {$x=0$}
                    child {node {$\exists z$} 
                        child {node [label=below:$(0\,1\,0)$] {$z=0$} edge from parent [dashed]}
                        child {node [label=below:$(0\,1\,1)$] {$z=1$}
                        edge from parent [dashed]}
                    }
                    edge from parent [red]
                }
                child {node {$x=1$}
                    child {node {$\exists z$}
                        child {node [label=below:$(1\,1\,0)$] {$z=0$}
                        edge from parent [dashed, black]}
                        child {node [label=below:\textcolor{green}{$(1\,1\,1)$}] {$z=1$}
                        edge from parent [dashed]}
                    }
                    edge from parent [green]
                }
                edge from parent [solid]
            } edge from parent [dashed]
        };
    \end{tikzpicture}
    Figure 1: Visual representation of $\exists y \forall x \exists z \predicate$
    
    $ $
    \end{center}
    }
    {
    \begin{center}\begin{tabular}{ccl}\
   $\exists y \forall x \exists z \predicate$ & $\equiv$ & $\forall x \exists z \predS(x,0,z) \lor \forall x \exists z \predS(x,1,z)$ \\ \\
    
    & $\equiv$ & $(\exists z \predS(0,0,z) \land \exists z \predS(1,0,z)) \lor (\exists z \predS(0,1,z) \land \exists z \predS(1,1,z))$ \\ \\
    
    & $\equiv$ & $((\predS(0,0,0) \lor \predS(0,0,1)) \land (\predS(1,0,0) \lor \predS(1,0,1))) \hspace{0.1cm} \lor$ \\
    & & $((\predS(0,1,0) \lor \predS(0,1,1)) \land (\predS(1,1,0) \lor \predS(1,1,1)))$ \\ \\
    
    & $\equiv$ & $((\False \lor \True) \land (\True \lor \False)) \lor ((\False \lor \False) \land (\False \lor \True))$ \\ \\
    
    & $\equiv$ & $(\True \land \True) \lor (\False \land \True)$ \\ \\
    
    & $\equiv$ & $\True \lor \False$ \\ \\
    
    & $\equiv$ & $\True$ \\ 
    \end{tabular}\end{center}
    This shows that the sentence is true on the provided model.
    }

    \pagebreak

    (iv) 
    {
    Evaluate $\exists y \forall x \exists z \predicate$ on $\{(1,0,0), (0,1,0), (0,1,1)\}$
    } \\
    {
    \begin{center}\begin{tikzpicture}
    \node {$\exists y$} [
        level 1/.style={sibling distance=8cm},
        level 2/.style={sibling distance=4cm}, 
        level 3/.style={sibling distance=4cm},
        level 4/.style={sibling distance=2cm}
    ]
        child {node {$y=0$}
            child {node {$\forall x$}
                child {node {$x=0$}
                    child {node {$\exists z$}
                        child {node [label=below:$(0\,0\,0)$] {$z=0$}
                        edge from parent [dashed]}
                        child {node [label=below:$(0\,0\,1)$] {$z=1$}
                        edge from parent [dashed]}
                    }
                } 
                child {node {$x=1$}
                    child {node {$\exists z$}
                         child {node [label=below:\textcolor{green}{$(1\,0\,0)$}] {$z=0$}
                         edge from parent [dashed]}
                         child {node [label=below:$(1\,0\,1)$] {$z=1$}
                         edge from parent [dashed, black]}
                    }
                    edge from parent [green]
                }
                edge from parent [solid]
            } edge from parent [dashed, red]
        }
        child {node {$y=1$}
            child {node {$\forall x$}
                child {node {$x=0$}
                    child {node {$\exists z$} 
                        child {node [label=below:\textcolor{green}{$(0\,1\,0)$}] {$z=0$} edge from parent [dashed]}
                        child {node [label=below:\textcolor{green}{$(0\,1\,1)$}] {$z=1$}
                        edge from parent [dashed]}
                    }
                    edge from parent [green]
                }
                child {node {$x=1$}
                    child {node {$\exists z$}
                        child {node [label=below:$(1\,1\,0)$] {$z=0$}
                        edge from parent [dashed]}
                        child {node [label=below:$(1\,1\,1)$] {$z=1$}
                        edge from parent [dashed]}
                    }
                }
                edge from parent [solid]
            } edge from parent [dashed, red]
        };
    \end{tikzpicture}
    Figure 1: Visual representation of $\exists y \forall x \exists z \predicate$
    
    $ $
    \end{center}
    }
    {
    \begin{center}\begin{tabular}{ccl}\
    $\exists y \forall x \exists z \predicate$ & $\equiv$ & $\forall x \exists z \predS(x,0,z) \lor \forall x \exists z \predS(x,1,z)$ \\ \\
    
    & $\equiv$ & $(\exists z \predS(0,0,z) \land \exists z \predS(1,0,z)) \lor (\exists z \predS(0,1,z) \land \exists z \predS(1,1,z))$ \\ \\
    
    & $\equiv$ & $((\predS(0,0,0) \lor \predS(0,0,1)) \land (\predS(1,0,0) \lor \predS(1,0,1))) \hspace{0.1cm} \lor$ \\
    & & $((\predS(0,1,0) \lor \predS(0,1,1)) \land (\predS(1,1,0) \lor \predS(1,1,1)))$ \\ \\
    
    & $\equiv$ & $((\False \lor \False) \land (\True \lor \False)) \lor ((\True \lor \True) \land (\False \lor \False))$ \\ \\
    
    & $\equiv$ & $(\False \land \True) \lor (\True \land \False)$ \\ \\
    
    & $\equiv$ & $\False \lor \False$ \\ \\
    
    & $\equiv$ & $\False$ \\ 
    \end{tabular}\end{center}
    This shows that the sentence is false on the provided model.
    }  
\end{enumerate}

\pagebreak
\printbibliography

\end{document}
